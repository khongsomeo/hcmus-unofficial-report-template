\chapter{Hướng dẫn sử dụng}

(bỏ phần này đi khi sử dụng)

\section{Tổng quan}
Đây là một template dạng thesis. Nội dung template sẽ theo các phần
\begin{itemize}
\item Chương 1: Giới thiệu (Introduction)
\item Chương 2: Các công trình liên quan (Related works)
\item Chuơng 3: Phuơng pháp đề xuất (Proposed method)
\item Chương 4: Thực nghiệm (Experiments \& Discussions)
\item Chương 5: Tổng kết (Conclusion)
\end{itemize}

Nhóm tác giả sẽ cố gắng hoàn thiện cho giống với cấu trúc \& yêu cầu hình thức của một template thesis nhất có thể:

\begin{enumerate}
\item Trang bìa, mục lục, danh sách hình vẽ, danh sách bảng biểu, nằm trong các mục đánh số La Mã.
\item Phần nội dung chính (bắt đầu chương I trở về sau) đánh số Ả-rập.
\item Lề trang (đang hoàn thiện)
\item Các trang bổ sung (nhận xét phản biện, nhận xét hướng dẫn, ..etc)
\end{enumerate}


\section{Hình ảnh và bảng biểu}

\subsection{Hình ảnh}

Hình \ref{fig:hcmus-with-no-h} là ví dụ cho một figure không có tag \texttt{[H]} -- bị trôi lên đầu trang. Hình \ref{fig:hcmus-with-h} là ví dụ cho figure có tag \texttt{[H]} -- không bị trôi lên đầu trang mà ở nguyên vị trí được chèn (sau đoạn văn này).

\begin{figure}
\centering
\includegraphics[scale=.2]{img/hcmus-logo.png}
\caption{Logo HCMUS không có tag \texttt{[H]}}
\label{fig:hcmus-with-no-h}
\end{figure}

\begin{figure}[H]
\centering
\includegraphics[scale=.2]{img/hcmus-logo.png}
\caption{Logo HCMUS với tag \texttt{[H]}}
\label{fig:hcmus-with-h}
\end{figure}

Tất cả các hình đều được tự động cập nhật trên Danh sách hình vẽ. Nếu vẫn chưa hiện trên Danh sách hình vẽ, thử compile lại 1--2 lần.

\paragraph{Về hình ảnh sử dụng trong bài} nếu chôm hình từ sách báo, \textbf{phải cite hoặc gửi email xin phép tác giả nếu cần thiết}. Ngoài ra, cân nhắc sử dụng định dạng \texttt{.svg} để bảo toàn chất lượng, tuy nhiên hình ảnh sẽ responsive (đúng nghĩa -- xem bên dưới).

 

\paragraph{Một chiêu thức lạ: TikZ}



\subsection{Bảng biểu}

\section{Mã nguồn \& Thuật toán}

\section{Định lý, mệnh đề, ..etc}

\section{Sử dụng tài liệu tham khảo}

\subsection{BibTeX}

File BibTeX tài liệu tham khảo nằm ở đường dẫn \texttt{ref/ref.bib}. Sửa tên file \texttt{.bib} sẽ phải sửa lại nội dung file \texttt{ref.tex}.

Đây là ví dụ trích dẫn (cite) một tài liệu\cite{greenwade93, goossens93} bằng style \texttt{ieeetr}. Với style này, công trình nào được cite trước sẽ có số thứ tự trước (bạn tự kiểm tra điều này). Ngoài ra còn một số style bạn có thể thử, bạn xem trong \href{https://www.bibtex.com/bibliography-styles/}{link này} (cảnh báo: \textbf{rất nhiều!})

\subsection{Lưu ý khi cite tài liệu}

Nếu có một nội dung cần được cite, ưu tiên chọn nguồn theo thứ tự sau (số thứ tự càng thấp thì càng ưu tiên):

\begin{enumerate}
\item Book: sách chuyên khảo, sách đã xuất bản.
\item Thesis: luận văn, luận án đã công bố.
\item Journal: các tạp chí khoa học chuyên ngành.
\item Conference: các kỉ yếu của hội nghị, hội thảo.
\item ArXiV: các trang preprint.
\item Blog: các website, blog, tutorial trên mạng.
\end{enumerate}

\paragraph{Ví dụ:} giả sử bài báo \textit{Symbolic Discovery of Optimization Algorithms} bạn tìm thấy 2 nguồn:
\begin{itemize}
\item \href{https://arxiv.org/abs/2302.06675}{ArXiV}
\item \href{https://proceedings.neurips.cc/paper\_files/paper/2023/hash/9a39b4925e35cf447ccba8757137d84f-Abstract-Conference.html}{NeurIPS 2023 proceedings}
\end{itemize}

Khi đó \textbf{dùng citation export từ NeurIPS 2023 proceedings} -- như thế này\cite{chen2024symbolic}. \textit{Tìm nguồn bài báo ở đâu?} -- bỏ tên bài báo vào Google Scholar hoặc tra trên paperswithcode.

\paragraph{Hạn chế dùng nguồn từ ArXiV hoặc blog} nguyên nhân là ai cũng post bài lên blog được, còn ArXiV là nền tảng preprint -- ai cũng có thể post paper chưa được bình duyệt lên. Do vậy, ta ưu tiên cite từ các nguồn có bình duyệt để đảm bảo độ tin cậy của thông tin.

\paragraph{Cách dẫn nguồn tài liệu} nội dung trích từ tài liệu nên được để trong dấu ngoặc kép "" hoặc viết lại bằng lời văn của mình nhưng vẫn giữ nguyên ý của tác giả, \textbf{cả 2 trường hợp đều phải cite}. Ngoài ra, \textbf{số liệu bảng biểu, hình ảnh sử dụng lại từ tài liệu (một số trường hợp phải gửi email xin phép tác giả)} đều phải cite.