\section{Hình ảnh}
Hình ảnh được thể hiện như hình~\ref{fig:my_label}, lưu ý flag \texttt{[H]} để disable floating (hình được hiển thị đúng vị trí, không trôi lên đầu trang).
\begin{figure}%[H]
    \centering
    \includegraphics[scale=.4]{img/hcmus-logo.png}
    \caption{Hình ví dụ (logo HCMUS - updated 30/11/2022)}
    \label{fig:my_label}
\end{figure}

Hình~\ref{fig:my_label_with_H} cũng là hình ví dụ nhưng có tag \texttt{[H]}. Lưu ý là có tag \texttt{[H]} thì code ở đâu hình sẽ nằm ở đó, không quan trọng nội dung ít hay nhiều (trang giấy sẽ thừa 1 khúc như bạn thấy). Để hiểu hơn về positioning trong LaTeX, xin tham khảo \href{https://www.overleaf.com/learn/latex/Positioning_images_and_tables}{bài này}.

\begin{figure}[H]
    \centering
    \includegraphics[scale=.4]{img/hcmus-logo.png}
    \caption{Hình ví dụ (logo HCMUS - updated 30/11/2022)}
    \label{fig:my_label_with_H}
\end{figure}

\section{Bảng biểu}
Bảng biểu được thể hiện như bảng~\ref{tab:my_label}, lưu ý flag \texttt{[H]} để disable floating (bảng được hiển thị đúng vị trí, không trôi lên đầu trang). Bảng~\ref{tab:my_label} là một trường hợp không sử dụng tag \texttt{[H]} và bảng bị trôi tít lên đầu trang:
\begin{table}%[H]
\centering
\begin{tabular}{|l|l|}
\hline
\textbf{Tên con vật} & \textbf{Số chân} \\ \hline
Gà & 2 \\ \hline
Chó & 4 \\ \hline
Trần Hoàng Tử & 2 \\ \hline
\end{tabular}
\caption{Số chân của một số con vật}
\label{tab:my_label}
\end{table}

Bảng~\ref{tab:my_label_with_H_tag} thể hiện bảng biểu với tag \texttt{[H]}\footnote{Tương tự cách sử dụng tag \texttt{[H]} với hình}. Để không phải mất thời gian tuổi trẻ ngồi chỉnh table, xài \href{https://www.tablesgenerator.com}{https://www.tablesgenerator.com}.

\begin{table}[H]
\centering
\begin{tabular}{|l|l|}
\hline
\textbf{Tên con vật} & \textbf{Số chân} \\ \hline
Gà & 2 \\ \hline
Chó & 4 \\ \hline
Trần Hoàng Tử & 2 \\ \hline
\end{tabular}
\caption{Số chân của một số con vật}
\label{tab:my_label_with_H_tag}
\end{table}.